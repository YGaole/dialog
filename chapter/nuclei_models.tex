\chapter{Exactly solvable models of nuclei}
精确可解的模型在核的壳模型发展中扮演了重要的作用,促进我们对原子核配对性质的理解以及在几何和代数模型的背景下对集体核现象的描述。本文综述了核的精确可解模型,重点讨论了概念问题而非技术问题。
\section{Introduction}
原子核是一个多体系统,主要受复杂和有效的 in-mideum 相互作用所控制,因此具有丰富的谱学性质。这些效应包括闭合壳层附近核子的独立运动,关联的两核子对的形成,以及由许多核子的协同运动引起的振动和转动的集体效应。目前对所观察到的各种核激发态的理论描述有两种可能的微观方法作为出发点。自洽平均场方法从一个给定的核子的有效相互作用或能量泛函开始来构造平均核子场;这导致了我们对集体模型的描述是从构成一个给定核子的所有中子和质子之间的相互关联来开始的。另一方面,球形核壳模型包含了在某种封闭壳结构外中子和质子的所有可能的相互作用。这两种方法都使用数值算法,因此是计算密集型的。

本文综述了这两种方法中能被精确地求解的一类子模型,并在此过程中强调了与原子核中成对关联系统的性质以及集体运动模式有关的一些一般结果。完全可解的模型必须具有图解性质,虽然仅对特定的核有效。但它们可以作为使用数值方法建立更真实的模型研究核素图大区域(同位素系列或同中子素系列)数据的参考或“基准”。这里的重点是精确可解模型本身,而不是与数据的比较。本文最后提到的几本书讨论了精确可解模型的后一个方面。